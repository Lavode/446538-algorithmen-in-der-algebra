\documentclass[a4paper]{scrreprt}

% Uncomment to optimize for double-sided printing.
% \KOMAoptions{twoside}

% Set binding correction manually, if known.
% \KOMAoptions{BCOR=2cm}

% Localization options
\usepackage[english]{babel}
\usepackage[T1]{fontenc}
\usepackage[utf8]{inputenc}

% Sub figures
\usepackage{subcaption}

\usepackage{algpseudocode}

% Quotations
\usepackage{dirtytalk}

% Floats
\usepackage{float}

% Enhanced verbatim sections. We're mainly interested in
% \verbatiminput though.
\usepackage{verbatim}

% Automatically remove leading whitespace in lstlisting
\usepackage{lstautogobble}

% CSV to tables
\usepackage{csvsimple}

% PDF-compatible landscape mode.
% Makes PDF viewers show the page rotated by 90°.
\usepackage{pdflscape}

% Advanced tables
\usepackage{array}
\usepackage{tabularx}
\usepackage{longtable}

% Fancy tablerules
\usepackage{booktabs}

% Graphics
\usepackage{graphicx}

% Current time
\usepackage[useregional=numeric]{datetime2}

% Float barriers.
% Automatically add a FloatBarrier to each \section
\usepackage[section]{placeins}

% Custom header and footer
\usepackage{fancyhdr}

\usepackage{geometry}
\usepackage{layout}

% Math tools
\usepackage{mathtools}
% Math symbols
\usepackage{amsmath,amsfonts,amssymb}
\usepackage{amsthm}
% General symbols
\usepackage{stmaryrd}

% Utilities for quotations
\usepackage{csquotes}


% Bibliography
\usepackage[
  style=alphabetic,
  backend=biber, % Default backend, just listed for completness
  sorting=ynt % Sort by year, name, title
]{biblatex}
\addbibresource{references.bib}

\DeclarePairedDelimiter\abs{\lvert}{\rvert}
\DeclarePairedDelimiter\floor{\lfloor}{\rfloor}

% Bullet point
\newcommand{\tabitem}{~~\llap{\textbullet}~~}

\newtheorem{theorem}{Theorem}[section]
\newtheorem{lemma}[theorem]{Lemma}

\pagestyle{plain}
% \fancyhf{}
% \lhead{}
% \lfoot{}
% \rfoot{}
% 
% Source code & highlighting
\usepackage{listings}

% SI units
\usepackage[binary-units=true]{siunitx}
\DeclareSIUnit\cycles{cycles}

\newcommand{\lecture}{446538 - Algorithmen in der Algebra}
\newcommand{\series}{5}
% Convenience commands
\newcommand{\mailsubject}{\lecture - Series \series}
\newcommand{\maillink}[1]{\href{mailto:#1?subject=\mailsubject}
                               {#1}}

% Should use this command wherever the print date is mentioned.
\newcommand{\printdate}{\today}

\subject{\lecture}
\title{Series \series}

\author{Michael Senn \maillink{michael.senn@students.unibe.ch} --- 16-126-880}

\date{\printdate}

% Needs to be the last command in the preamble, for one reason or
% another. 
\usepackage{hyperref}

\begin{document}
\maketitle


\setcounter{chapter}{\numexpr \series - 1 \relax}

\chapter{Series \series}

\section{Kernel of $A \in \mathbb{Z}^{m \times n}$ is a lattice in $\mathbb{R}^n$}

Let $A \in \mathbb{Z}_{m \times n}$ be an $m \times n$ matrix over
$\mathbb{Z}$. Let $X \coloneqq \{x \in \mathbb{Z}^n : Ax = 0\}$. Note first
that, by its definition, $X = ker(A)$.

We first show that $X$ is closed with respect to adition. Let $x, x' \in
ker(A)$, then:
\begin{align*}
		A(x + x') & = Ax + Ax' \\
				  & = 0 \\
				  & \Rightarrow x + x' \in ker(A)
\end{align*}

Further $0 \in X$ and, if $x \in X$, then $-x \in X$ as $(-x)A = -(Ax) = 0$, so
$(X, +)$ is a subgroup of $\mathbb{R}^n$. We also know that $\mathbb{Z}^n$ is
discrete, so $X \subset \mathbb{Z}^n$ is discrete as well. Thus, $(X, +)$ is a
lattice over $\mathbb{R^n}$.

Lastly by the rank nullity theorem we know that $\dim \mathbb{Z}^n = \dim Im(A)
+ \dim Ker(A)$, so the dimension of the lattice is given as $\dim Ker(A) = \dim
\mathbb{Z}^n - \dim Im(A) = n - \dim Im(A)$.

\section{Kernel of $A \in \mathbb{Z}^{m \times n}$ is a lattice in $(\mathbb{Z} / N\mathbb{Z})^m$}

\section{Subgroup of $(\mathbb{R}, +)$ generated by $(1, \sqrt{2})$ is not a lattice}

Let $X \coloneqq \mathbb{Z} + \sqrt{2} \mathbb{Z} = \{a + b \sqrt{2} : a, b \in
\mathbb{Z}\}$. We intend to show that the subgroup $(X, +)$ of $\mathbb{R}$ is
not a lattice.

Note first that $X$ is closed under addition and multiplication. Let $a + b
\sqrt{2} \in X$, $c + d \sqrt{2} \in X$. Then:
\begin{align*}
		(a + b \sqrt{2}) + (c + d \sqrt{2}) & = (a + c) + (b + d) \sqrt{2} \in X \\
		(a + b \sqrt{2}) \cdot (c + d \sqrt{2}) & = ac + ad \sqrt{2} + bc \sqrt{2} + 2bd \\
												& = (ac + 2bd) + (ad + bc) \sqrt{2} \in X
\end{align*}

Now consider the sequence $(a_n) = \left(1 - \sqrt{2}\right)^n$. As $\left|1 -
\sqrt{2}\right| < 1$, $(a_n)$ converges towards $0$. Further since $X$ is
closed under addition and multiplication, $(a_n) \subset X$. However, $0 \in
X$, so $X$ is not discrete, and $(X, +)$ is thus not a lattice.

\section{Base reduction using Lagrange-Gauss algorithm}

We intend to reduce the basis $p = (1, 1414), q = (0, 1000)$ of a lattice in
$\mathbb{R}^2$ by hand, using the Lagrange-Gauss algorithm. In the following
table, values of $\mu_i$ refer to the $\mu$ used in the subtraction to get the
values $b_{1, i+1}, b_{2, i+1}$ of the next iteration. \\

\begin{tabular}{lllll}
		\toprule
		$i$ & $b_{1, i}$ & $b_{2, i}$ & $\mu_{i}$ & $||b_{1, i} - \mu b_{2, i}||$ \\
		\midrule
		1 & $(1, 1414)$ & $(0, 1000)$ & $1$ & $\approx 400$ \\
		2 & $(0, 1000)$ & $(1, 414)$  & $2$ & $\approx 170$ \\
		3 & $(1, 414)$  & $(-2, 172)$ & $2$ & $\approx 70$ \\
		4 & $(-2, 172)$ & $(5, 70)$   & $2$ & $\approx 35$ \\
		5 & $(5, 70)$   & $(-12, 32)$ & $2$ & $\approx 30$ \\
		6 & $(-12, 32)$ & $(29, 6)$   & $0$ & - \\
		\bottomrule
\end{tabular}\\

Thus $(-12, 32), (29, 6)$ is a reduced basis, as both $||(-12, 32) - (29, 6)||
> ||(-12, 32)||$ as well as $||(-12, 32) + (29, 6)|| > ||(-12, 32)||$.

\section{Finding approximation of $\pi^{-1}$ using LLL}

Using LLL with randomly generated bases of $\mathbb{Z}^2$, various
approximations of $\pi^{-1}$ were found. Of those with a denominator and
numerator less than 30, the best was $7/22$ with an absolute error of
$|\pi^{-1} - 7/22| < 0.00013$. Turns out we've rediscovered the well-known
approximation $\pi \approx 22/7$ by brute force.

\printbibliography{}

\end{document}
