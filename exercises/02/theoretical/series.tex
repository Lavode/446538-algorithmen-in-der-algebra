\documentclass[a4paper]{scrreprt}

% Uncomment to optimize for double-sided printing.
% \KOMAoptions{twoside}

% Set binding correction manually, if known.
% \KOMAoptions{BCOR=2cm}

% Localization options
\usepackage[english]{babel}
\usepackage[T1]{fontenc}
\usepackage[utf8]{inputenc}

% Sub figures
\usepackage{subcaption}

% Quotations
\usepackage{dirtytalk}

% Floats
\usepackage{float}

% Enhanced verbatim sections. We're mainly interested in
% \verbatiminput though.
\usepackage{verbatim}

% Automatically remove leading whitespace in lstlisting
\usepackage{lstautogobble}

% CSV to tables
\usepackage{csvsimple}

% PDF-compatible landscape mode.
% Makes PDF viewers show the page rotated by 90°.
\usepackage{pdflscape}

% Advanced tables
\usepackage{array}
\usepackage{tabularx}
\usepackage{longtable}

% Fancy tablerules
\usepackage{booktabs}

% Graphics
\usepackage{graphicx}

% Current time
\usepackage[useregional=numeric]{datetime2}

% Float barriers.
% Automatically add a FloatBarrier to each \section
\usepackage[section]{placeins}

% Custom header and footer
\usepackage{fancyhdr}

\usepackage{geometry}
\usepackage{layout}

% Math tools
\usepackage{mathtools}
% Math symbols
\usepackage{amsmath,amsfonts,amssymb}
\usepackage{amsthm}
% General symbols
\usepackage{stmaryrd}

% Utilities for quotations
\usepackage{csquotes}

% Bibliography
\usepackage[
  style=alphabetic,
  backend=biber, % Default backend, just listed for completness
  sorting=ynt % Sort by year, name, title
]{biblatex}
\addbibresource{references.bib}

\DeclarePairedDelimiter\abs{\lvert}{\rvert}
\DeclarePairedDelimiter\floor{\lfloor}{\rfloor}

% Bullet point
\newcommand{\tabitem}{~~\llap{\textbullet}~~}

\pagestyle{plain}
% \fancyhf{}
% \lhead{}
% \lfoot{}
% \rfoot{}
% 
% Source code & highlighting
\usepackage{listings}

% SI units
\usepackage[binary-units=true]{siunitx}
\DeclareSIUnit\cycles{cycles}

\newcommand{\lecture}{446538 - Algorithmen in der Algebra}
\newcommand{\series}{2}
% Convenience commands
\newcommand{\mailsubject}{\lecture - Series \series}
\newcommand{\maillink}[1]{\href{mailto:#1?subject=\mailsubject}
                               {#1}}

% Should use this command wherever the print date is mentioned.
\newcommand{\printdate}{\today}

\subject{\lecture}
\title{Series \series}

\author{Michael Senn \maillink{michael.senn@students.unibe.ch} --- 16-126-880}

\date{\printdate}

% Needs to be the last command in the preamble, for one reason or
% another. 
\usepackage{hyperref}

\begin{document}
\maketitle


\setcounter{chapter}{\numexpr \series - 1 \relax}

\chapter{Series \series}

\section{Extended euclidean algorithm}

\section{Solving $x^2 = 1$ in ring of integers modulo $n = pq$}

Given $n = pq$ where $p, q$ are prime, we intend to solve $x^2 = 1$ in
$\mathbb{Z} / n\mathbb{Z}$. First note that:
\begin{align*}
		x^2 = 1 \pmod{pq} \Leftrightarrow x \equiv 1 \pmod{pq} \lor x \equiv (-1) \pmod {pq}
\end{align*}

We can solve these equivalence relations by means of the CRT, by instead
solving and recombining the following four sets of equivalence relations:
\begin{align*}
		x & \equiv 1 & \pmod{p} \\
		x & \equiv 1 & \pmod{q} \\
		\\
		x & \equiv 1 & \pmod{p} \\
		x & \equiv (-1) \equiv q-1 & \pmod{q} \\
		\\
		x & \equiv (-1) \equiv p-1 & \pmod{p} \\
		x & \equiv 1 & \pmod{q} \\
		\\
		x & \equiv (-1) \equiv p-1 & \pmod{p} \\
		x & \equiv (-1) \equiv q-1 & \pmod{q}
\end{align*}

Given $p = 101$, $q = 97$ we will first use Bézout's identity to find integers
$m_1, m_2$ such that:
\begin{align*}
		m_1 p + m_2 q = gcd(p, q)
\end{align*}

Using the extended euclidean algorithm, one finds $m_1 = -24$, $m_2 = 25$.
Indeed $-24 \cdot 101 + 25 \cdot 97 = 1$.

Then, by the CRT, solutions for $x$ can be found as $x = a_1 m_2 q + a_2 m_1 p$,
where $a_1$ is the residue $\bmod p$, $a_2$ the residue $\bmod q$. Thus:
\begin{align*}
		1 \cdot 25 \cdot 97 + 1 \cdot -24 \cdot 101 & = 1 \\
		1 \cdot 25 \cdot 97 + 96 \cdot -24 \cdot 101 = -230279 & \equiv 4849 \pmod{pq} \\
		100 \cdot 25 \cdot 97 + 1 \cdot -24 \cdot 101 = 240076 & \equiv 4948 \pmod {pq} \\
		100 \cdot 25 \cdot 97 + 96 \cdot -24 \cdot 101 & = 9796
\end{align*}

Which are the four solutions to $x^2 = 1$ in $\mathbb{Z} / pq \mathbb{Z}$.

\section{Implementation of ffficient modular exponentiation and Miller-Rabin primality test}

Handed in separately.

\end{document}
