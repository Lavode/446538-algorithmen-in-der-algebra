\documentclass[a4paper]{scrreprt}

% Uncomment to optimize for double-sided printing.
% \KOMAoptions{twoside}

% Set binding correction manually, if known.
% \KOMAoptions{BCOR=2cm}

% Localization options
\usepackage[english]{babel}
\usepackage[T1]{fontenc}
\usepackage[utf8]{inputenc}

% Sub figures
\usepackage{subcaption}

\usepackage{algpseudocode}

% Quotations
\usepackage{dirtytalk}

% Floats
\usepackage{float}

% Enhanced verbatim sections. We're mainly interested in
% \verbatiminput though.
\usepackage{verbatim}

% Automatically remove leading whitespace in lstlisting
\usepackage{lstautogobble}

% CSV to tables
\usepackage{csvsimple}

% PDF-compatible landscape mode.
% Makes PDF viewers show the page rotated by 90°.
\usepackage{pdflscape}

% Advanced tables
\usepackage{array}
\usepackage{tabularx}
\usepackage{longtable}

% Fancy tablerules
\usepackage{booktabs}

% Graphics
\usepackage{graphicx}

% Current time
\usepackage[useregional=numeric]{datetime2}

% Float barriers.
% Automatically add a FloatBarrier to each \section
\usepackage[section]{placeins}

% Custom header and footer
\usepackage{fancyhdr}

\usepackage{geometry}
\usepackage{layout}

% Math tools
\usepackage{mathtools}
% Math symbols
\usepackage{amsmath,amsfonts,amssymb}
\usepackage{amsthm}
% General symbols
\usepackage{stmaryrd}

% Utilities for quotations
\usepackage{csquotes}


% Bibliography
\usepackage[
  style=alphabetic,
  backend=biber, % Default backend, just listed for completness
  sorting=ynt % Sort by year, name, title
]{biblatex}
\addbibresource{references.bib}

\DeclarePairedDelimiter\abs{\lvert}{\rvert}
\DeclarePairedDelimiter\floor{\lfloor}{\rfloor}

% Bullet point
\newcommand{\tabitem}{~~\llap{\textbullet}~~}

\pagestyle{plain}
% \fancyhf{}
% \lhead{}
% \lfoot{}
% \rfoot{}
% 
% Source code & highlighting
\usepackage{listings}

% SI units
\usepackage[binary-units=true]{siunitx}
\DeclareSIUnit\cycles{cycles}

\newcommand{\lecture}{446538 - Algorithmen in der Algebra}
\newcommand{\series}{2}
% Convenience commands
\newcommand{\mailsubject}{\lecture - Series \series}
\newcommand{\maillink}[1]{\href{mailto:#1?subject=\mailsubject}
                               {#1}}

% Should use this command wherever the print date is mentioned.
\newcommand{\printdate}{\today}

\subject{\lecture}
\title{Series \series}

\author{Michael Senn \maillink{michael.senn@students.unibe.ch} --- 16-126-880}

\date{\printdate}

% Needs to be the last command in the preamble, for one reason or
% another. 
\usepackage{hyperref}

\begin{document}
\maketitle


\setcounter{chapter}{\numexpr \series - 1 \relax}

\chapter{Series \series}

\section{Extended binary euclidean algorithm}

We intend to extend the presented binary euclidean algorithm shown in the
lecture, to also produce the coefficients of Bézout's identity, that is to find
integers $x, y$ such that:
\[
		xa + yb = gcd(a, b)
\]

The following is an incomplete (and rather WIP) approach to the problem, based
on an algorithm presented in algorithm 14.61 of the handbook of applied
cryptography \autocite{menezesHandbookAppliedCryptography1997}. It includes
only those parts I fully understood in the time spent on the exercise.

\subsection{Extracting common factors of two at the very beginning}

First note that removing common powers of two can be done at the very
beginning. Consider the following at the very beginning of the presented
algorithm:

\begin{algorithmic}
 	\State $k = 1$

 	\While{$a, b$ even}
 			\State $a = a / 2$
 			\State $b = b / 2$
 			\State $k = k * 2$
 	\EndWhile
\end{algorithmic}

As seen in the lecture, $gcd(a, b) = 2^k \cdot gcd(a / 2^k, b / 2^k)$, so this
does not affect the resulting GCD as long as the factor of $2^k$ is multiplied
back at the end. Similarly, if $xa + yb = gcd(a, b)$ then $xa / 2^k + yb / 2^k
= 2^{-k} (xa + yb) = 2^{-k} gcd(a, b)$ so pulling out a common factor $2^k$
does not affect the coefficients of Bézout's identity either.

This simplifies the main loop, as at most one of $a$ or $b$ can be even in that
case.

\subsection{Main loop}

We now introduce four auxilliary variables $p_i, q_i, r_i, s_i$ with $p_1, q_1,
r_1, s_1 = 1, 0, 0, 1$. Further let $u_1, v_1 = a, b$. Our invariant will be
that, at the end of each loop $i$, the following holds:
\begin{align*}
		p_i \cdot a + q_i \cdot b = u_i \\
		r_i \cdot a + s_i \cdot b = v_i
\end{align*}

Clearly for the first iteration $i = 1$ this holds as $p_1 \cdot a + q_1 \cdot
b = a = u_1$, and $r_1 \cdot a + s_1 \cdot b = b = v_1$.

The main loop will then iterate until $u_i = 0$. The cases handled within
are as follows:

\subsubsection{If $u_i$ even, $v_i$ odd, $p_i, q_i$ even}

As in the lecture, let $u_{i+1} = u_i / 2$. Further let $p_{i+1} = p_{i} / 2$
and $q_{i+1} = q_{i} / 2$ and $r_{i+1} = r_i, s_{i+1} = s_i$. Then:
\begin{align*}
		p_{i+1} a + q_{i+1} b & = \frac{p_i}{2} a + \frac{q_i}{2} b \\
							  & = \frac{1}{2} (p_i a + q_i b) \\
							  & = \frac{1}{2} u_i \\
							  & = u_{i+1}
\end{align*}

so both invariants hold.

\subsubsection{If $v_i$ even, $u_i$ odd, $r_i, s_i$ even}

As in the lecture, let $v_{i+1} = v_i / 2$. Further let $r_{i+1} = r_{i} / 2$
and $s_{i+1} = s_{i} / 2$ and $p_{i+1} = p_i$ and $q_{i+1} = q_i$. Then:
\begin{align*}
		r_{i+1} a + s_{i+1} b & = \frac{r_i}{2} a + \frac{s_i}{2} b \\
							  & = \frac{1}{2} (r_i a + s_i b) \\
							  & = \frac{1}{2} v_i \\
							  & = v_{i+1}
\end{align*}

so both invariants hold.

\section{Solving $x^2 = 1$ in ring of integers modulo $n = pq$}

Given $n = pq$ where $p, q$ are prime, we intend to solve $x^2 = 1$ in
$\mathbb{Z} / n\mathbb{Z}$. First note that:
\begin{align*}
		x^2 = 1 \pmod{pq} \Leftrightarrow x \equiv 1 \pmod{pq} \lor x \equiv (-1) \pmod {pq}
\end{align*}

We can solve these equivalence relations by means of the CRT, by instead
solving and recombining the following four sets of equivalence relations:
\begin{align*}
		x & \equiv 1 & \pmod{p} \\
		x & \equiv 1 & \pmod{q} \\
		\\
		x & \equiv 1 & \pmod{p} \\
		x & \equiv (-1) \equiv q-1 & \pmod{q} \\
		\\
		x & \equiv (-1) \equiv p-1 & \pmod{p} \\
		x & \equiv 1 & \pmod{q} \\
		\\
		x & \equiv (-1) \equiv p-1 & \pmod{p} \\
		x & \equiv (-1) \equiv q-1 & \pmod{q}
\end{align*}

Given $p = 101$, $q = 97$ we will first use Bézout's identity to find integers
$m_1, m_2$ such that:
\begin{align*}
		m_1 p + m_2 q = gcd(p, q)
\end{align*}

Using the extended euclidean algorithm, one finds $m_1 = -24$, $m_2 = 25$.
Indeed $-24 \cdot 101 + 25 \cdot 97 = 1$.

Then, by the CRT, solutions for $x$ can be found as $x = a_1 m_2 q + a_2 m_1 p$,
where $a_1$ is the residue $\bmod p$, $a_2$ the residue $\bmod q$. Thus:
\begin{align*}
		1 \cdot 25 \cdot 97 + 1 \cdot -24 \cdot 101 & = 1 \\
		1 \cdot 25 \cdot 97 + 96 \cdot -24 \cdot 101 = -230279 & \equiv 4849 \pmod{pq} \\
		100 \cdot 25 \cdot 97 + 1 \cdot -24 \cdot 101 = 240076 & \equiv 4948 \pmod {pq} \\
		100 \cdot 25 \cdot 97 + 96 \cdot -24 \cdot 101 & = 9796
\end{align*}

Which are the four solutions to $x^2 = 1$ in $\mathbb{Z} / pq \mathbb{Z}$.

\section{Implementation of efficient modular exponentiation and Miller-Rabin primality test}

Handed in separately.

\printbibliography{}

\end{document}
