\documentclass[a4paper]{scrreprt}

% Uncomment to optimize for double-sided printing.
% \KOMAoptions{twoside}

% Set binding correction manually, if known.
% \KOMAoptions{BCOR=2cm}

% Localization options
\usepackage[english]{babel}
\usepackage[T1]{fontenc}
\usepackage[utf8]{inputenc}

% Sub figures
\usepackage{subcaption}

\usepackage{algpseudocode}

% Quotations
\usepackage{dirtytalk}

% Floats
\usepackage{float}

% Enhanced verbatim sections. We're mainly interested in
% \verbatiminput though.
\usepackage{verbatim}

% Automatically remove leading whitespace in lstlisting
\usepackage{lstautogobble}

% CSV to tables
\usepackage{csvsimple}

% PDF-compatible landscape mode.
% Makes PDF viewers show the page rotated by 90°.
\usepackage{pdflscape}

% Advanced tables
\usepackage{array}
\usepackage{tabularx}
\usepackage{longtable}

% Fancy tablerules
\usepackage{booktabs}

% Graphics
\usepackage{graphicx}

% Current time
\usepackage[useregional=numeric]{datetime2}

% Float barriers.
% Automatically add a FloatBarrier to each \section
\usepackage[section]{placeins}

% Custom header and footer
\usepackage{fancyhdr}

\usepackage{geometry}
\usepackage{layout}

% Math tools
\usepackage{mathtools}
% Math symbols
\usepackage{amsmath,amsfonts,amssymb}
\usepackage{amsthm}
% General symbols
\usepackage{stmaryrd}

% Utilities for quotations
\usepackage{csquotes}


% Bibliography
\usepackage[
  style=alphabetic,
  backend=biber, % Default backend, just listed for completness
  sorting=ynt % Sort by year, name, title
]{biblatex}
\addbibresource{references.bib}

\DeclarePairedDelimiter\abs{\lvert}{\rvert}
\DeclarePairedDelimiter\floor{\lfloor}{\rfloor}

% Bullet point
\newcommand{\tabitem}{~~\llap{\textbullet}~~}

\newtheorem{theorem}{Theorem}[section]
\newtheorem{lemma}[theorem]{Lemma}

\pagestyle{plain}
% \fancyhf{}
% \lhead{}
% \lfoot{}
% \rfoot{}
% 
% Source code & highlighting
\usepackage{listings}

% SI units
\usepackage[binary-units=true]{siunitx}
\DeclareSIUnit\cycles{cycles}

\newcommand{\lecture}{446538 - Algorithmen in der Algebra}
\newcommand{\series}{4}
% Convenience commands
\newcommand{\mailsubject}{\lecture - Series \series}
\newcommand{\maillink}[1]{\href{mailto:#1?subject=\mailsubject}
                               {#1}}

% Should use this command wherever the print date is mentioned.
\newcommand{\printdate}{\today}

\subject{\lecture}
\title{Series \series}

\author{Michael Senn \maillink{michael.senn@students.unibe.ch} --- 16-126-880}

\date{\printdate}

% Needs to be the last command in the preamble, for one reason or
% another. 
\usepackage{hyperref}

\begin{document}
\maketitle


\setcounter{chapter}{\numexpr \series - 1 \relax}

\chapter{Series \series}

\section{Reducibility of $x^4 + 1$ in $\mathbb{Z}[x]$ and $\mathbb{F}_p[x]$}

\subsection{$x^4 + 1$ is irreducible in $\mathbb{Z}[x]$}

Note first that $x^4 + 1$ has no roots in $\mathbb{Z}[x]$, as there are no
solutions to $x^4 = 1$ in $\mathbb{Z}$. Thus if $x^4 + 1$ was reducible, it
would factor into two polynomials of degree two:
\begin{align*}
		x^4 + 1 & = (x^2 + ax + b) (x^2 + a'x + b') \\
				& = x^4 + (a' + a) x^3 + (b' + aa' + b) x^2 + (ab' + ba') x + (bb')
\end{align*}

This yields four linear equations:
\begin{align*}
		a' + a & = 0 && \text{(I)} \\
		b' + aa' + b & = 0 && \text{(II)} \\
		ab' + ba' & = 0 && \text{(III)} \\
		bb' & = 1 && \text{(IV)}
\end{align*}

Which we can solve:
\begin{align*}
		\text{(I)} & \Rightarrow a' = -a && \text{(V)} \\
		\text{(III)} & \Rightarrow a'b + ab' 
		               \stackrel{\text{(IV)}}{=} ab' - ab = 0 
		               \Rightarrow b = b' && \text{(VI)} \\
		\text{(IV)} & \Rightarrow bb' 
		               \stackrel{\text{(VI)}}{=} b^2 
				       \Rightarrow |b| = 1 \\
		\text{(II)} & \Rightarrow b' + aa' + b 
		              \stackrel{\text{(VI)}}{=} 2b + aa' 
					  \stackrel{\text{(V)}}{=} 2b - a^2
					  \Rightarrow a^2 = 2
\end{align*}

However $a^2 = 2$ has no solution in $\mathbb{Z}$, so no such factorization can
exist, thus the polynomial $x^4 + 1$ is irreducible in $\mathbb{Z}[x]$.

\subsection{$x^4 + 1$ is reducible in $\mathbb{F}_p$ for every prime $p$}

For $p = 2$, $(x^2 + 1)^2 = x^4 + 2 x^2 + 1 \equiv x^4 + 1$, so the polynomial
is reducible. Let now $p$ be an odd prime. We will first prove the following:

\begin{lemma}
		If $x \in \mathbb{Z}$ is odd, then exactly one of $(x - 1)$, $(x + 1)$
		is a multiple of four, while the other is a multiple of two but not of
		four.
\end{lemma}

\begin{proof}
		As $x$ is odd, both $x - 1$ as well as $x + 1$ are even. Thus either $x
		- 1 \equiv 0 \pmod{4}$ or $x - 1 \equiv 2 \pmod{4}$. If $x - 1 \equiv 0
		\pmod{4}$ then $x + 1 = x - 1 + 2 \equiv 0 + 2 = 2 \pmod{4}$. Similarly
		if $x - 1 \equiv 2 \pmod{4}$ then $x + 1 \equiv 0 \pmod{4}$.
\end{proof}

Now consider the multiplicative group of the field $\mathbb{F}_{p^2}$. As $p$
is prime, this group's order is $p^2 - 1 = (p - 1) (p + 1)$. As seen above we
know that $(p^2 - 1) \equiv 0 \pmod{8}$. Since this group is finite cyclic,
there exists an element $\alpha$ of order $8$. (Indeed if $g$ generates the
whole group, then $g^\frac{p^2 - 1}{8}$ will be such an element).

Then, $\alpha$ will be a root of the polynomial $x^8 - 1 = (x^4 - 1)(x^4 + 1)$.
It will also be a root of $x^4 + 1$ but not of $x^4 - 1$, as that one's roots
will have an order dividing four. Thus $\mathbb{F}_{p^2}$ is the splitting
field of $x^4 + 1$ over $\mathbb{F}_p$. 

Now, if $x^4 + 1$ was irreducible, then $4 | [\mathbb{F}_{p^2} /
\mathbb{F}_p]$. However due to the multiplicativity of degrees of field
extensions we know that $[\mathbb{F}_{p^2} / \mathbb{F}_p] = 2$, which is a
contradiction. Hence, $x^4 + 1$ is reducible in $\mathbb{F}_p$.

\section{Computational complexity of ``FaktorI'' algorithm is not polynomial in degree of polynomial}

Given a polynomial $f$ of degree $n$, then the factorization in $GF(p)$ for $p$
prime using Berlekamp might return up to $n$ linear factors $f = g_1 g_2 \ldots
g_n \mod{p}$. But then there are $O(2^n)$ subsets $I$ of indices of
$\tilde{g}_i$ such that the sum of their degrees does not exceed $\lfloor n/2
\rfloor$.

If the polynomial happens to be irreducible over $\mathbb{Z}$, then all of
these possible subsets must be enumerated, leading to an exponential runtime
complexity.

\printbibliography{}

\end{document}
