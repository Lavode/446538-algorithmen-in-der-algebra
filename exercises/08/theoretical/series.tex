\documentclass[a4paper]{scrreprt}

% Uncomment to optimize for double-sided printing.
% \KOMAoptions{twoside}

% Set binding correction manually, if known.
% \KOMAoptions{BCOR=2cm}

% Localization options
\usepackage[english]{babel}
\usepackage[T1]{fontenc}
\usepackage[utf8]{inputenc}

% Sub figures
\usepackage{subcaption}

\usepackage{algpseudocode}

% Quotations
\usepackage{dirtytalk}

% Floats
\usepackage{float}

% Enhanced verbatim sections. We're mainly interested in
% \verbatiminput though.
\usepackage{verbatim}

% Automatically remove leading whitespace in lstlisting
\usepackage{lstautogobble}

% CSV to tables
\usepackage{csvsimple}

% PDF-compatible landscape mode.
% Makes PDF viewers show the page rotated by 90°.
\usepackage{pdflscape}

% Advanced tables
\usepackage{array}
\usepackage{tabularx}
\usepackage{longtable}

% Fancy tablerules
\usepackage{booktabs}

% Graphics
\usepackage{graphicx}

% Current time
\usepackage[useregional=numeric]{datetime2}

% Float barriers.
% Automatically add a FloatBarrier to each \section
\usepackage[section]{placeins}

% Custom header and footer
\usepackage{fancyhdr}

\usepackage{geometry}
\usepackage{layout}

% Math tools
\usepackage{mathtools}
% Math symbols
\usepackage{amsmath,amsfonts,amssymb}
\usepackage{amsthm}
% General symbols
\usepackage{stmaryrd}

% Utilities for quotations
\usepackage{csquotes}


% Bibliography
\usepackage[
  style=alphabetic,
  backend=biber, % Default backend, just listed for completness
  sorting=ynt % Sort by year, name, title
]{biblatex}
\addbibresource{references.bib}

\DeclarePairedDelimiter\abs{\lvert}{\rvert}
\DeclarePairedDelimiter\floor{\lfloor}{\rfloor}

% Bullet point
\newcommand{\tabitem}{~~\llap{\textbullet}~~}

\newtheorem{theorem}{Theorem}[section]
\newtheorem{lemma}[theorem]{Lemma}

\pagestyle{plain}
% \fancyhf{}
% \lhead{}
% \lfoot{}
% \rfoot{}
% 
% Source code & highlighting
\usepackage{listings}

% SI units
\usepackage[binary-units=true]{siunitx}
\DeclareSIUnit\cycles{cycles}

\newcommand{\lecture}{446538 - Algorithmen in der Algebra}
\newcommand{\series}{8}
% Convenience commands
\newcommand{\mailsubject}{\lecture - Series \series}
\newcommand{\maillink}[1]{\href{mailto:#1?subject=\mailsubject}
                               {#1}}

% Should use this command wherever the print date is mentioned.
\newcommand{\printdate}{\today}

\subject{\lecture}
\title{Series \series}

\author{Michael Senn \maillink{michael.senn@students.unibe.ch} --- 16-126-880}

\date{\printdate}

% Needs to be the last command in the preamble, for one reason or
% another. 
\usepackage{hyperref}

\begin{document}
\maketitle


\setcounter{chapter}{\numexpr \series - 1 \relax}

\chapter{Series \series}

\section{Uniqueness of monomial order in one variable}

We aim to show that there is only one distinct monomial order in one
variable. Let $M = \{1, x^1, x^2, \ldots\}$ be the set of all monomials in one
variable. Let $M_i \coloneqq \{x^i, x^{i+1}, \ldots \}$ be $M$ exlcuding
monomials of order less than $i$.

First note that, for any monomial order $\leq$, $1$ is the minimal element of
$M$. To see this assume that a different element $x^a \in M$ was the minimal
element. But then $x^a \leq 1$, so $x^{(k+1)a} \leq x^{ka}$ for all $k \in
\mathbb{N}$, so $M$ would not have a minimal element, so $\leq$ would not be a
well-order.

As above note that, for any monomial order $\leq$, $x^n$ is the minimal element
of $M_n$. Assume that instead a diferent element $x^a \in M_n, a > n$ was the
minimal element. But then $x^{2a} \leq x^{a + n}$, yet $a > n \Rightarrow 2a >
a+n$, so again $\leq$ would not have a minimal element in $M_n$.

Thus for any monomial order $\leq$, $x^0$ is the minimal element in $M$, $x^1$
in $M \setminus\{x^0\}$, $x^n$ in $M \setminus \{x^0, x^1, \ldots, x^{n-1}\}$ and
so on. Thus there is only one monomial order in one variable.

\section{Example of a monomial order}

Consider the monomial order in two variables defined by:
\[
		x_1^a x_2^b \leq x_1^c x_2^d \Leftrightarrow a + b \sqrt{2} \leq c + d \sqrt{2}
\]

Let $M$ be the set of monomials in two variables. Let $x_1^a x_2^b, x_1^c
x_2^d, x_1^e x_1^f \in M$.

\subsection{$\leq$ defines a total order}

\subsubsection{$\leq$ is reflexive}

$a + b \sqrt{2} \leq a + b \sqrt{2}$, so $\leq$ is reflexive.

\subsubsection{$\leq$ is transitive}

Assume $x_1^a x_1^b \leq x_1^c x_2^d$ and $x_1^c x_2^d \leq x_1^e x_2^f$. Thus
$a + b \sqrt{2} \leq c + d \sqrt{2}$ and $c + d \sqrt{2} \leq e + f \sqrt{2}$.
Then $a + b \sqrt{2} \leq e + f \sqrt{2}$ so $\leq$ is transitive.

\subsubsection{$\leq$ is antisymmetric}

Assume $x_1^a x_2^b \leq x_1^c x_2^d$ and $x_1^c x_2^d \leq x_1^a x_2^b$. Thus
$a + b \sqrt{2} \leq c + d \sqrt{2} \leq a + b \sqrt{2}$. Note that $\sqrt{2}$
is irrational, so $x \sqrt{2}, x \in \mathbb{N}$ is irrational too. Thus $a = c
\land b = d$ so $\leq$ is antisymmetric.

\subsubsection{$\leq$ is total}

Lastly for all $a, b, c, d \in \mathbb{N}$, the comparison $a + b \sqrt{2}
\overset{?}{\leq} c + d \sqrt{2}$ is meaningful (in e.g. $\mathbb{R}$), so the
order is total.

\subsection{$\leq$ defines a well-order}

As seen above, $x \sqrt{2}$ for $x \in \mathbb{N}$ is always an irrational
number. Thus, two elements $a + b \sqrt{2}$, $c + d \sqrt{2}$ will be equal if
and only if $a = c \land b = d$.

Let $M' \subset M$ be a non-empty subset. Consider the set $A \coloneqq \{a + b
\sqrt{2} : x_1^a x_2^b \in M'\}$. $A$ is a subset of the reals with no
duplicate elements, so the standard total order on the reals ensures existence
of a unique minimal element $a + b \sqrt{2} \in A$. Then $x_1^a x_2^b \in M'$
is the unique minimal element of $M'$, so $\leq$ is a well-order.

\subsection{$\leq$ defines a monomial order}

Let $a, b, c, d, e, f \in \mathbb{N}$ with $a + b \sqrt{2} \leq c + d
\sqrt{2}$. Then:

\begin{align*}
		(a + e) + (b + d) \sqrt{2} & = (a + b \sqrt{2}) + (e + f \sqrt{2}) \\
								   & \leq (c + d \sqrt{2}) + (e + f \sqrt{2}) \\
								   & = (c + e) + (d + f) \sqrt{2}
\end{align*}

So:
\begin{align*}
		x_1^a x_2^b x_1^e x_2^f & = x_1^{a+e} x_2^{b+f} \\
								& \leq x_1^{c+e} x_2^{d+f} \\
								& = x_1^c x_2^d x_1^e x_2^f
\end{align*}

So $\leq$ is a monomial order.

\section{Non-examples of monomial orders}

\section{Monomial orders in two variables}

\printbibliography{}

\end{document}
