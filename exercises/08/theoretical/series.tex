\documentclass[a4paper]{scrreprt}

% Uncomment to optimize for double-sided printing.
% \KOMAoptions{twoside}

% Set binding correction manually, if known.
% \KOMAoptions{BCOR=2cm}

% Localization options
\usepackage[english]{babel}
\usepackage[T1]{fontenc}
\usepackage[utf8]{inputenc}

% Sub figures
\usepackage{subcaption}

\usepackage{algpseudocode}

% Quotations
\usepackage{dirtytalk}

% Floats
\usepackage{float}

% Enhanced verbatim sections. We're mainly interested in
% \verbatiminput though.
\usepackage{verbatim}

% Automatically remove leading whitespace in lstlisting
\usepackage{lstautogobble}

% CSV to tables
\usepackage{csvsimple}

% PDF-compatible landscape mode.
% Makes PDF viewers show the page rotated by 90°.
\usepackage{pdflscape}

% Advanced tables
\usepackage{array}
\usepackage{tabularx}
\usepackage{longtable}

% Fancy tablerules
\usepackage{booktabs}

% Graphics
\usepackage{graphicx}

% Current time
\usepackage[useregional=numeric]{datetime2}

% Float barriers.
% Automatically add a FloatBarrier to each \section
\usepackage[section]{placeins}

% Custom header and footer
\usepackage{fancyhdr}

\usepackage{geometry}
\usepackage{layout}

% Math tools
\usepackage{mathtools}
% Math symbols
\usepackage{amsmath,amsfonts,amssymb}
\usepackage{amsthm}
% General symbols
\usepackage{stmaryrd}

% Utilities for quotations
\usepackage{csquotes}


% Bibliography
\usepackage[
  style=alphabetic,
  backend=biber, % Default backend, just listed for completness
  sorting=ynt % Sort by year, name, title
]{biblatex}
\addbibresource{references.bib}

\DeclarePairedDelimiter\abs{\lvert}{\rvert}
\DeclarePairedDelimiter\floor{\lfloor}{\rfloor}

% Bullet point
\newcommand{\tabitem}{~~\llap{\textbullet}~~}

\newtheorem{theorem}{Theorem}[section]
\newtheorem{lemma}[theorem]{Lemma}

\pagestyle{plain}
% \fancyhf{}
% \lhead{}
% \lfoot{}
% \rfoot{}
% 
% Source code & highlighting
\usepackage{listings}

% SI units
\usepackage[binary-units=true]{siunitx}
\DeclareSIUnit\cycles{cycles}

\newcommand{\lecture}{446538 - Algorithmen in der Algebra}
\newcommand{\series}{8}
% Convenience commands
\newcommand{\mailsubject}{\lecture - Series \series}
\newcommand{\maillink}[1]{\href{mailto:#1?subject=\mailsubject}
                               {#1}}

% Should use this command wherever the print date is mentioned.
\newcommand{\printdate}{\today}

\subject{\lecture}
\title{Series \series}

\author{Michael Senn \maillink{michael.senn@students.unibe.ch} --- 16-126-880}

\date{\printdate}

% Needs to be the last command in the preamble, for one reason or
% another. 
\usepackage{hyperref}

\begin{document}
\maketitle


\setcounter{chapter}{\numexpr \series - 1 \relax}

\chapter{Series \series}

\section{Uniqueness of monomial order in one variable}

Let $M = \{1, x^1, x^2, \ldots\}$ be the set of all monomials in one variable.
Let $(M_i)_{i \in \mathbb{N}} \coloneqq \{x^i, x^{i+1}, \ldots \}$ be the
subset of $M$ exlcuding monomials of order less than $i$.

First note that, for any monomial order $\leq$ on $M$, $1$ is the least element
of $M$. To see this assume that a different element $x^a \in M$ was the least
element. But then $x^a \leq 1$, so $x^{(k+1)a} \leq x^{ka}$ for all $k \in
\mathbb{N}$, so $M$ would not have a least element, so $\leq$ would not be a
well-order.

Now note that, for any monomial order $\leq$ on $M_n$, $x^n$ is the least
element of $M_n$. Assume that instead a diferent element $x^a \in M_n, a > n$
was the least element. But then $x^{2a} \leq x^{a + n}$, yet $a > n \Rightarrow
2a > a+n$, so again $\leq$ would not have a least element in $M_n$.

Thus for any monomial order $\leq$, $x^0$ is the least element in $M$, $x^1$
in $M \setminus\{x^0\}$, $x^n$ in $M \setminus \{x^0, x^1, \ldots, x^{n-1}\}$ and
so on. So there is only one monomial order in one variable.

\section{Example of a monomial order}
\label{sec:monomial_order}

Let $M$ be the set of monomials in two variables. Let $x_1^a x_2^b, x_1^c
x_2^d, x_1^e x_1^f \in M$. Consider the binary relation $\leq$ over $M$,
defined by:
\[
		x_1^a x_2^b \leq x_1^c x_2^d \Leftrightarrow a + b \sqrt{2} \leq c + d \sqrt{2}
\]

\subsection{$\leq$ defines a total order}

\subsubsection{Reflexivity}

$a + b \sqrt{2} \leq a + b \sqrt{2}$, so $\leq$ is reflexive.

\subsubsection{Transitivity}

Assume $x_1^a x_1^b \leq x_1^c x_2^d$ and $x_1^c x_2^d \leq x_1^e x_2^f$. Thus
$a + b \sqrt{2} \leq c + d \sqrt{2}$ and $c + d \sqrt{2} \leq e + f \sqrt{2}$.
Then $a + b \sqrt{2} \leq e + f \sqrt{2}$ so $\leq$ is transitive.

\subsubsection{Antisymmetry}

Assume $x_1^a x_2^b \leq x_1^c x_2^d$ and $x_1^c x_2^d \leq x_1^a x_2^b$. Thus
$a + b \sqrt{2} \leq c + d \sqrt{2} \leq a + b \sqrt{2}$. Note that $\sqrt{2}$
is irrational, so $x \sqrt{2}, x \in \mathbb{N}$ is irrational too. If follows
that $a = c \land b = d$ so $\leq$ is antisymmetric.

\subsubsection{Totality}

Lastly for all $a, b, c, d \in \mathbb{N}$, the comparison $a + b \sqrt{2}
\overset{?}{\leq} c + d \sqrt{2}$ is meaningful (in e.g. $\mathbb{R}$), so the
order is total.

\subsection{Well-order}

As above, $x \sqrt{2}$ for $x \in \mathbb{N}$ is always irrational. Thus, two
elements $a + b \sqrt{2}$, $c + d \sqrt{2}$ will be equal if and only if $a = c
\land b = d$.

Let $M' \subset M$ be a non-empty subset. Consider the set $A \coloneqq \{a + b
\sqrt{2} : x_1^a x_2^b \in M'\}$. $A$ is a subset of the reals with no
duplicate elements, so the standard total order on the reals ensures existence
of a unique least element $a + b \sqrt{2} \in A$. Then $x_1^a x_2^b \in M'$
is the unique least element of $M'$, so $\leq$ is a well-order.

\subsection{Monomial order}

Let $a, b, c, d, e, f \in \mathbb{N}$ with $a + b \sqrt{2} \leq c + d
\sqrt{2}$. Then:

\begin{align*}
		(a + e) + (b + d) \sqrt{2} & = (a + b \sqrt{2}) + (e + f \sqrt{2}) \\
								   & \leq (c + d \sqrt{2}) + (e + f \sqrt{2}) \\
								   & = (c + e) + (d + f) \sqrt{2}
\end{align*}

So:
\begin{align*}
		x_1^a x_2^b x_1^e x_2^f & = x_1^{a+e} x_2^{b+f} \\
								& \leq x_1^{c+e} x_2^{d+f} \\
								& = x_1^c x_2^d x_1^e x_2^f
\end{align*}

So $\leq$ is a monomial order.

\section{Non-examples of monomial orders}

Let $M$ be the set of monomials in two variables. Let $x_1^a x_2^b, x_1^c
x_2^d, x_1^e x_1^f \in M$.

\subsection{Not a well-order}

Consider the binary relation $\leq$ over $M$, defined by:
\[
		x_1^a x_2^b \leq x_1^c x_2^d \Leftrightarrow a - b \sqrt{2} \leq c - d \sqrt{2}
\]

Note that $(-n \sqrt{2})_{n \in \mathbb{N}}$ is monotonically decreasing and
divergent, so the set $\{-n \sqrt{2} : n \in \mathbb{N}\} \subset \mathbb{R}$
has no least element with the usual total order over the reals. As such the
binary relation above is no well-order, and thus also no monomial order.

\subsection{Not antisymmetric}


Consider the binary relation $\leq$ over $M$, defined by:
\[
		x_1^a x_2^b \leq x_1^c x_2^d \Leftrightarrow a + b \frac{17}{433} \leq c + d \frac{17}{433}
\]

Let $u \coloneqq x_1^0 x_2^{433}, v \coloneqq x_1^{17} x_2^0 \in M$. Clearly $u \neq v$, yet:
\begin{align*}
		0 + 433 \frac{17}{433} = 17 = 17 + 0 \frac{17}{433} \\
		\Rightarrow u \leq v \land v \leq u
\end{align*}

So the relation is not antisymmetric, and thus also no monomial order.


\section{Monomial orders in two variables}

Let be a positive irrational number. Consider the binary relation $\leq_r$ on
the set of monomials in two variables $M$, defined by:

\[
		x_1^a x_2^b \leq x_1^c x_2^d \Leftrightarrow a + br \leq c + dr
\]

Each of these binary relations $\leq_r$ can be shown to define a monomial order
by generalizing the proof for $r = \sqrt{2}$ in section \ref{sec:monomial_order}.

As the set of positive irrational number is uncountably infinite, there are
also an uncountably infinite number of monomial orders in two variables.

\section{Complexity of algorithm for division of multivariate polynomials}

Consider the algorithm presented in the lecture. Let $k$ be the number of
divisors $h_1, \ldots, h_k$. Let $f$ be the dividend. Let $N$ be the maximum
number of terms in either of the $h_i$ or of $f$.

In each iteration of the loop we must --- at most --- consider each pair $(h_i,
c x^\alpha)$, where $c x^\alpha$ is a term of $f$, to find out whether there
exists an $h_i$ such that $LM(h_i) | x^\alpha$. There are at most $kN$ such pairs.

Further as was shown, the number of monomials in the remainder $r$ is strictly
decreasing inbetween any two iterations, ensuring that the algorithm
terminates. As such there are at most $N$ total iterations.

Thus the runtime of the algorithm is indeed polynomial in $kN$. The specific
monomial order may influence the exact complexity insofar as it influences
which divisions take place, but will not affect the asymptotical complexity.


\printbibliography{}

\end{document}
