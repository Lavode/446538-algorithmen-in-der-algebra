\documentclass[a4paper]{scrreprt}

% Uncomment to optimize for double-sided printing.
% \KOMAoptions{twoside}

% Set binding correction manually, if known.
% \KOMAoptions{BCOR=2cm}

% Localization options
\usepackage[english]{babel}
\usepackage[T1]{fontenc}
\usepackage[utf8]{inputenc}

% Sub figures
\usepackage{subcaption}

% Quotations
\usepackage{dirtytalk}

% Floats
\usepackage{float}

% Enhanced verbatim sections. We're mainly interested in
% \verbatiminput though.
\usepackage{verbatim}

% Automatically remove leading whitespace in lstlisting
\usepackage{lstautogobble}

% CSV to tables
\usepackage{csvsimple}

% PDF-compatible landscape mode.
% Makes PDF viewers show the page rotated by 90°.
\usepackage{pdflscape}

% Advanced tables
\usepackage{array}
\usepackage{tabularx}
\usepackage{longtable}

% Fancy tablerules
\usepackage{booktabs}

% Graphics
\usepackage{graphicx}

% Current time
\usepackage[useregional=numeric]{datetime2}

% Float barriers.
% Automatically add a FloatBarrier to each \section
\usepackage[section]{placeins}

% Custom header and footer
\usepackage{fancyhdr}

\usepackage{geometry}
\usepackage{layout}

% Math tools
\usepackage{mathtools}
% Math symbols
\usepackage{amsmath,amsfonts,amssymb}
\usepackage{amsthm}
% General symbols
\usepackage{stmaryrd}

% Utilities for quotations
\usepackage{csquotes}

% Bibliography
\usepackage[
  style=alphabetic,
  backend=biber, % Default backend, just listed for completness
  sorting=ynt % Sort by year, name, title
]{biblatex}
\addbibresource{references.bib}

\DeclarePairedDelimiter\abs{\lvert}{\rvert}
\DeclarePairedDelimiter\floor{\lfloor}{\rfloor}

% Bullet point
\newcommand{\tabitem}{~~\llap{\textbullet}~~}

\pagestyle{plain}
% \fancyhf{}
% \lhead{}
% \lfoot{}
% \rfoot{}
% 
% Source code & highlighting
\usepackage{listings}

% SI units
\usepackage[binary-units=true]{siunitx}
\DeclareSIUnit\cycles{cycles}

\newcommand{\lecture}{446538 - Algorithmen in der Algebra}
\newcommand{\series}{1}
% Convenience commands
\newcommand{\mailsubject}{\lecture - Series \series}
\newcommand{\maillink}[1]{\href{mailto:#1?subject=\mailsubject}
                               {#1}}

% Should use this command wherever the print date is mentioned.
\newcommand{\printdate}{\today}

\subject{\lecture}
\title{Series \series}

\author{Michael Senn \maillink{michael.senn@students.unibe.ch} --- 16-126-880}

\date{\printdate}

% Needs to be the last command in the preamble, for one reason or
% another. 
\usepackage{hyperref}

\begin{document}
\maketitle


\setcounter{chapter}{\numexpr \series - 1 \relax}

\chapter{Series \series}

\section{Implementation of Karatsuba multiplication in Sagemath}

Handed in separately.

\section{Complexity analysis of Karatsuba multiplication}

Given the intermediary product $(x_{hi} + x_{lo}) (y_{hi} + y_{lo})$ is a
product of two factors of size up to $n/2 + 1$, a more accurate estimate of the
runtime complexity of Karatsuba multiplication might thus be, for some constant
$C$:
\begin{align*}
		T(2n) \leq 2 T(n) + T(n+1) + Cn
\end{align*}

Note however that we can get an upper bound for $T(n+1) - T(n)$ using e.g. the
known complexity of the gradeschool algorithm:
\begin{align*}
		T(n + 1) - T(n) \leq (n+1)^2 - n^2 = 2n + 1 \leq 3n
\end{align*}

Inserting this into the estimate above:
\begin{align*}
		T(2n) & \leq 2 T(n) + T(n+1) + Cn \\
			  & = 3 T(n) + (T(n+1) - T(n)) + Cn \\
			  & \leq 3 T(n) + 3n + Cn \\
			  & = 3 T(n) + n(3 + C)
\end{align*}

We can thus pick a constant $\hat{C} \coloneqq C + 3 = T(2) + 3$, and then the
following estimate will hold once more --- albeit for a larger constant
$\hat{C}$:
\begin{align*}
		T(2n) \leq 3 T(n) + \hat{C} n
\end{align*}

It is likely that this bound could be made more tight if desired.

\section{Complexity analysis of generalised Karatsuba multiplication}

Not handed in.

\section{Determinant of Vandermonde matrix}

We aim to prove the following by induction:
\[
		\det M(x_0, \ldots, x_k) = \prod_{0 \leq i < j \leq k} (x_j - x_i)
\]

\begin{proof}

For the induction basis, observe that for $k = 1$, the Vandermonde matrix
$M(x_0, x_1)$ has the following form and determinant:
\begin{align*}
		M(x_0, x_1) & = \begin{bmatrix}
				1 & x_0 \\
				1 & x _1
		\end{bmatrix} \\
		\det M(x_0, x_1) & = x_1 - x_0 = \prod_{0 \leq i < j \leq 1} (x_j - x_i)
\end{align*}

For the induction step, note that for $k + 1$ the $(k+2) \times (k+2)$
Vandermonde matrix takes the following form:
\begin{align*}
		M(x_0, x_1, \ldots, x_{k+1}) = \begin{bmatrix}
				1 & x_0  & x_0^2 & \ldots & x_0^{k+1} \\
				1 & x_1  & x_1^2 & \ldots & x_1^{k+1} \\
				\vdots & \vdots & \vdots & \vdots \\
				1 & x_{k+1}  & x_{k+1}^2 & \ldots & x_{k+1}^{k+1}
		\end{bmatrix}
\end{align*}

Subtract $x_0 M_{i-1}$ from each column $M_{i}, 1 \leq i \leq k+1$:
\begin{align*}
		\det M(x_0, x_1, \ldots, x_{k+1}) = \det \begin{bmatrix}
				1 & 0  & 0 & \ldots & 0 \\
				1 & x_1 - x_0 & x_1^2 - x_0 x_1 & \ldots & x_1^{k+1} - x_0 x_1^{k} \\
				\vdots & \vdots & \vdots & \vdots \\
				1 & x_{k+1} - x_0  & x_{k+1}^2 - x_0 x_{k+1} & \ldots & x_{k+1}^{k+1} - x_0 x_{k+1}^{k}
		\end{bmatrix}
\end{align*}

By the properties of block matrices, the determinant of the resultant matrix is
equal to the determinant of its lower-right $(k+1) \times (k+1)$ block:
\begin{align*}
		\det M(x_0, x_1, \ldots, x_{k+1}) = \det \begin{bmatrix}
				x_1 - x_0 & x_1^2 - x_0 x_1 & \ldots & x_1^{k+1} - x_0 x_1^{k} \\
				\vdots & \vdots & \vdots \\
				x_{k+1} - x_0  & x_{k+1}^2 - x_0 x_{k+1} & \ldots & x_{k+1}^{k+1} - x_0 x_{k+1}^{k}
		\end{bmatrix}
\end{align*}

We then pull a common factor $(x_{i+1} - x_0)$ from every $i$-th row out of the
determinant:
\begin{align*}
		\det M(x_0, x_1, \ldots, x_{k+1}) = (x_1 - x_0) (x_2 - x_0) \ldots (x_{k+1} - x_0) 
		\det \begin{bmatrix}
				1 & x_1 & \ldots & x_1^{k} \\
				\vdots & \vdots & \vdots \\
				1 & x_{k+1} & \ldots & x_{k+1}^{k}
		\end{bmatrix}
\end{align*}

This matrix is now a Vandermonde matrix (with indices shifted by one)
once more. We thus use the base case and simplify:
\begin{align*}
\det M(x_0, x_1, \ldots, x_{k+1}) & = (x_1 - x_0) (x_2 - x_0) \ldots (x_{k+1} - x_0)
		                              \det \begin{bmatrix}
		                              		1 & x_1 & \ldots & x_1^{k} \\
		                              		\vdots & \vdots & \vdots \\
		                              		1 & x_{k+1} & \ldots & x_{k+1}^{k}
		                              \end{bmatrix} \\
								  & = \left(\prod_{1 \leq j \leq k+1} (x_j - x_0)\right) 
								      \left(\prod_{1 \leq i < j \leq {k+1}} (x_j - x_i)\right) \\
								  & = \prod_{0 \leq i < j \leq k+1} (x_j - x_i)
\end{align*}

\end{proof}

\end{document}
